\documentclass[12pt,a4paper]{report}

\usepackage[utf8]{inputenc}
\usepackage[T1]{fontenc}
\usepackage[spanish]{babel}
\usepackage{geometry}
\usepackage{graphicx}
\usepackage{float}
\usepackage{booktabs}
\usepackage{siunitx}
\usepackage{amsmath}
\usepackage{hyperref}
\usepackage{caption}
\usepackage{subcaption}
\usepackage{enumitem}
\usepackage{xcolor}

\geometry{margin=2.5cm}

\sisetup{
  output-decimal-marker = {.},
  per-mode = symbol,
  detect-all = true
}

\hypersetup{
  colorlinks=true,
  linkcolor=blue,
  citecolor=blue,
  urlcolor=blue
}

\title{\textbf{Simulación Biomecánica de la Córnea Humana mediante Elementos Finitos}\\
\large Modelado y Análisis en ANSYS Workbench para Simulador de Palpación Corneal}
\author{Pablo Barceló, Paco Bustos, Javier Rivero}
\date{\today}

\begin{document}
\maketitle
\tableofcontents
\clearpage

\chapter*{Resumen}
\addcontentsline{toc}{chapter}{Resumen}
Este documento constituye la memoria técnica de la componente de \textbf{simulación mediante Método de Elementos Finitos (FEM)} desarrollada en \textbf{ANSYS Workbench/Mechanical} para un proyecto de simulador quirúrgico de palpación corneal. El trabajo se centra en el modelado biomecánico de la córnea humana, incluyendo la definición geométrica, caracterización de materiales, configuración del mallado, aplicación de condiciones de contorno y análisis de la respuesta mecánica del tejido ante cargas de indentación.

Se describe el modelo FEM multicapa desarrollado, que incluye el estroma corneal (particionado en regiones anterior y posterior) y la capa de Bowman, así como las simplificaciones adoptadas para obtener un modelo computacionalmente eficiente y físicamente representativo. Los resultados incluyen campos de desplazamiento, tensiones y deformaciones que permiten caracterizar la rigidez aparente del tejido y establecer relaciones fuerza-desplazamiento.

El modelo desarrollado constituye la base para la futura integración en un entorno de simulación interactivo en tiempo real, proporcionando los datos biomecánicos necesarios para el feedback háptico en entrenadores quirúrgicos oftalmológicos.

\vspace{0.5cm}
\noindent\textbf{Palabras clave:} Biomecánica corneal, Elementos Finitos, ANSYS, Simulación quirúrgica, Palpación, Indentación.

\chapter{Introducción}

\section{Contexto y motivación}
La cirugía oftalmológica requiere una destreza manual excepcional y un conocimiento profundo de las propiedades biomecánicas de los tejidos oculares. La córnea, como principal superficie refractiva del ojo, es objeto de múltiples intervenciones quirúrgicas (refractivas, transplantes, crosslinking, etc.) que demandan un entrenamiento riguroso por parte de los cirujanos. La palpación corneal, técnica fundamental en la evaluación intraoperatoria de la rigidez del tejido, constituye una habilidad crítica que tradicionalmente se adquiere mediante práctica en modelos animales o directamente en pacientes.

El desarrollo de simuladores quirúrgicos basados en realidad virtual y feedback háptico representa una alternativa ética, segura y reproducible para el entrenamiento de cirujanos oftalmólogos. Estos sistemas permiten la práctica repetida sin riesgo para pacientes, proporcionando métricas objetivas de desempeño y acelerando la curva de aprendizaje~\cite{Blackburn2019Review}.

Para que un simulador quirúrgico sea efectivo, debe reproducir fielmente el comportamiento mecánico del tejido real. Esto requiere modelos biomecánicos precisos que capturen la respuesta de la córnea ante diferentes tipos de interacción mecánica. El Método de Elementos Finitos (FEM) se ha establecido como la técnica de referencia para modelar el comportamiento de tejidos blandos bajo cargas complejas~\cite{Pang2024Review}.

\section{Objetivos del proyecto}
El objetivo principal de este trabajo es desarrollar y validar un \textbf{modelo de elementos finitos de la córnea humana} que permita simular la respuesta mecánica del tejido ante cargas de palpación/indentación. Los objetivos específicos son:

\begin{enumerate}[leftmargin=1.2cm]
  \item Diseñar una geometría simplificada pero representativa de la córnea humana, incluyendo sus principales estructuras (estroma y capa de Bowman).
  \item Caracterizar las propiedades mecánicas de los diferentes componentes corneales a partir de datos experimentales de la literatura.
  \item Implementar el modelo FEM en ANSYS Workbench/Mechanical con configuraciones apropiadas de mallado, contactos y condiciones de contorno.
  \item Simular escenarios de palpación corneal mediante aplicación de cargas controladas y obtener campos de desplazamiento, tensiones y deformaciones.
  \item Establecer relaciones fuerza-desplazamiento que caractericen la rigidez aparente del modelo y que puedan utilizarse para calibrar sistemas hápticos.
  \item Validar el modelo mediante comparación con datos experimentales y estudios previos de la literatura.
  \item Identificar limitaciones del modelo actual y proponer mejoras para iteraciones futuras.
\end{enumerate}

\section{Alcance del trabajo}
Este documento cubre exclusivamente la \textbf{componente de simulación mediante elementos finitos} del proyecto completo de simulador quirúrgico. El alcance incluye:

\begin{itemize}[leftmargin=1.2cm]
  \item Diseño conceptual y construcción geométrica del modelo corneal en ANSYS.
  \item Definición de materiales y propiedades mecánicas basadas en evidencia experimental.
  \item Configuración completa del análisis FEM (mallado, elementos, contactos, cargas).
  \item Ejecución de simulaciones estáticas y extracción de resultados.
  \item Análisis de convergencia y verificación de la calidad de las soluciones.
  \item Documentación detallada de la metodología para reproducibilidad.
\end{itemize}

\noindent El alcance \textbf{no incluye}:
\begin{itemize}[leftmargin=1.2cm]
  \item Integración con motores de simulación interactivos (Unity, Unreal Engine).
  \item Implementación de dispositivos hápticos y su calibración.
  \item Desarrollo de interfaces de usuario para el simulador.
  \item Validación clínica o estudios con usuarios.
\end{itemize}

Estos aspectos complementarios serían desarrollados por otros equipos del proyecto en fases posteriores.

\section{Estructura del documento}
El resto del documento se organiza como sigue:

\begin{itemize}[leftmargin=1.2cm]
  \item \textbf{Capítulo 2} presenta el estado del arte en simuladores quirúrgicos oftalmológicos y modelado biomecánico de la córnea.
  \item \textbf{Capítulo 3} detalla la metodología empleada en el desarrollo del modelo FEM, incluyendo geometría, materiales, mallado y configuración del análisis.
  \item \textbf{Capítulo 4} presenta los resultados obtenidos de las simulaciones.
  \item \textbf{Capítulo 5} discute las conclusiones, limitaciones y líneas futuras de trabajo.
\end{itemize}

\chapter{Estado del Arte}

\section{Simuladores quirúrgicos en oftalmología}

\subsection{Evolución histórica y clasificación}
Los simuladores quirúrgicos en medicina han evolucionado desde modelos físicos simples hasta sistemas complejos de realidad virtual con feedback háptico. En oftalmología, esta evolución ha sido particularmente notable debido a la naturaleza delicada de los procedimientos y la dificultad de obtener material de práctica~\cite{Blackburn2019Review}.

Los simuladores oftalmológicos actuales pueden clasificarse en:
\begin{itemize}[leftmargin=1.2cm]
  \item \textbf{Simuladores físicos}: Ojos sintéticos o modelos anatómicos que replican propiedades mecánicas mediante materiales especializados.
  \item \textbf{Simuladores de realidad virtual}: Sistemas basados en gráficos 3D que recrean visualmente el procedimiento quirúrgico.
  \item \textbf{Simuladores con feedback háptico}: Combinan realidad virtual con dispositivos que reproducen las fuerzas y sensaciones táctiles del procedimiento real.
  \item \textbf{Simuladores híbridos}: Integran componentes físicos con elementos virtuales y hápticos.
\end{itemize}

\subsection{Simuladores para cirugía de cataratas}
La cirugía de cataratas ha sido el procedimiento oftalmológico más ampliamente simulado. Sistemas como el Eyesi Surgical (VRmagic) proporcionan entrenamiento en facoemulsificación con métricas objetivas de rendimiento. Estudios han demostrado que el entrenamiento en simuladores reduce el tiempo quirúrgico y mejora los resultados en cirujanos noveles~\cite{Shih2017SciRep}.

\subsection{Simuladores para cirugía refractiva}
Los procedimientos refractivos (LASIK, PRK) requieren precisión micrométrica y han motivado el desarrollo de simuladores específicos. La importancia de modelar correctamente la biomecánica corneal en estos casos es crítica, ya que la ablación laser modifica permanentemente las propiedades mecánicas del tejido~\cite{Kling2014PLOS}.

\subsection{Simuladores para procedimientos de retina}
La cirugía vitreorretiniana presenta desafíos únicos por la delicadeza de las estructuras y las fuerzas involucradas. Simuladores como el EyeSi Vitreoretinal han demostrado efectividad en el entrenamiento de maniobras básicas y avanzadas.

\subsection{Necesidad de modelos biomecánicos precisos}
Un elemento común a todos los simuladores avanzados es la necesidad de modelos biomecánicos que capturen fielmente el comportamiento mecánico de los tejidos. Sin estos modelos, el feedback háptico carece de realismo y el valor del entrenamiento se ve comprometido~\cite{Karimi2019IndentEye}.

\section{Biomecánica corneal}

\subsection{Estructura anatómica de la córnea}
La córnea es un tejido avascular, transparente y altamente organizado compuesto por cinco capas principales:

\begin{itemize}[leftmargin=1.2cm]
  \item \textbf{Epitelio}: Capa celular más externa (50 \si{\micro\meter}), función protectora.
  \item \textbf{Capa de Bowman}: Membrana acelular (8-12 \si{\micro\meter}), contribuye a la rigidez superficial.
  \item \textbf{Estroma}: Constituye el 90\% del espesor corneal (450-500 \si{\micro\meter}), compuesto por lamelas de colágeno altamente organizadas.
  \item \textbf{Membrana de Descemet}: Membrana basal del endotelio (10 \si{\micro\meter}).
  \item \textbf{Endotelio}: Monocapa celular (5 \si{\micro\meter}), regula la hidratación estromal.
\end{itemize}

El estroma es el principal contribuyente a las propiedades mecánicas de la córnea debido a su espesor y su estructura fibrilar de colágeno~\cite{Dias2013}.

\subsection{Propiedades mecánicas del tejido corneal}
La córnea exhibe un comportamiento mecánico complejo caracterizado por:

\begin{itemize}[leftmargin=1.2cm]
  \item \textbf{No linealidad}: La relación tensión-deformación no es lineal, mostrando rigidización a grandes deformaciones.
  \item \textbf{Anisotropía}: Las propiedades varían según la dirección debido a la orientación preferencial de las fibras de colágeno.
  \item \textbf{Viscoelasticidad}: Exhibe comportamiento dependiente del tiempo (creep, relajación de tensiones).
  \item \textbf{Incompresibilidad}: Como tejido hidratado, es aproximadamente incompresible ($\nu \approx 0.49$).
\end{itemize}

\subsection{Técnicas experimentales de caracterización}
Diversas técnicas se han empleado para caracterizar las propiedades mecánicas corneales:

\begin{itemize}[leftmargin=1.2cm]
  \item \textbf{Ensayos de tracción uniaxial}: Determinan curvas tensión-deformación en tiras de córnea.
  \item \textbf{Ensayos de inflación}: Aplican presión interna para evaluar la respuesta global del tejido.
  \item \textbf{Nanoindentación/AFM}: Permiten medir propiedades locales a escala microscópica~\cite{Last2012Micron,Dias2013}.
  \item \textbf{Elastografía}: Técnicas de imagen no invasivas para mapear rigidez in vivo.
  \item \textbf{Air-puff}: Sistemas clínicos que deforman la córnea mediante pulsos de aire y analizan su respuesta dinámica~\cite{Kling2014PLOS}.
\end{itemize}

\subsection{Variabilidad de propiedades mecánicas}
La literatura reporta rangos amplios para el módulo de Young del estroma corneal (0.05-5 MPa), dependiendo de:
\begin{itemize}[leftmargin=1.2cm]
  \item Técnica experimental empleada
  \item Condiciones de hidratación del tejido
  \item Región corneal medida (central vs periférica)
  \item Profundidad estromal (anterior vs posterior)
  \item Estado patológico (queratocono, crosslinking, etc.)
\end{itemize}

Estudios recientes mediante nanoindentación han establecido que el estroma anterior es aproximadamente 3 veces más rígido que el posterior~\cite{Dias2013,Last2012Micron}.

\section{Modelado mediante elementos finitos de la córnea}

\subsection{Revisión de modelos FEM corneales}
El modelado FEM de la córnea ha experimentado un desarrollo considerable en las últimas dos décadas. Pang et al.~\cite{Pang2024Review} proporcionan una revisión exhaustiva de los enfoques utilizados, identificando tendencias y desafíos.

Los primeros modelos empleaban geometrías simplificadas (esféricas) y materiales lineales elásticos. La evolución hacia modelos más sofisticados ha incorporado:
\begin{itemize}[leftmargin=1.2cm]
  \item Geometrías personalizadas basadas en tomografía
  \item Modelos constitutivos hiperelásticos (Neo-Hooke, Mooney-Rivlin, Ogden)
  \item Anisotropía mediante modelos de refuerzo con fibras (HGO)~\cite{HGO2000}
  \item Viscoelasticidad mediante modelos de Prony o Maxwell
  \item Acoplamiento con presión intraocular (IOP)
\end{itemize}

\subsection{Aplicaciones de modelos FEM}
Los modelos FEM se han aplicado a múltiples escenarios clínicos:

\paragraph{Cirugía refractiva}
Shih et al.~\cite{Shih2017SciRep} desarrollaron modelos para predecir la respuesta biomecánica tras LASIK y PRK, demostrando concentraciones de tensiones en las zonas de ablación que pueden correlacionarse con complicaciones postoperatorias.

\paragraph{Tonometría}
Qin et al.~\cite{Qin2022Frontiers} emplearon análisis paramétrico mediante FEM para determinar propiedades biomecánicas corneales a partir de mediciones de tonometría de aplanación, estableciendo que la rigidez corneal influye significativamente en las lecturas de IOP.

\paragraph{Queratocono y crosslinking}
Modelos FEM han sido utilizados para simular la progresión del queratocono y evaluar la efectividad del crosslinking corneal como tratamiento de rigidización del tejido~\cite{Blackburn2019Review}.

\paragraph{Indentación y palpación}
Ren et al.~\cite{Ren2021Indent} y Karimi et al.~\cite{Karimi2019IndentEye} han desarrollado modelos específicos para simular ensayos de indentación, combinando FEM con datos experimentales para validar propiedades mecánicas efectivas bajo condiciones fisiológicas.

\subsection{Condiciones de contorno y su influencia}
Gómez et al.~\cite{Gomez2024BC} demostraron que las condiciones de contorno tienen un impacto significativo en la respuesta del modelo FEM. La restricción del limbo corneal y el modelado del globo ocular completo vs. córnea aislada produce diferencias notables en los campos de desplazamiento y tensiones.

\subsection{Modelos multicapa y gradientes de propiedades}
Towler et al.~\cite{Towler2023Heliyon} desarrollaron modelos con propiedades mecánicas localizadas, considerando variaciones espaciales dentro del estroma. Wu et al.~\cite{Wu2024Striae} analizaron el papel de las estrías estromales en el comportamiento mecánico corneal.

\subsection{Optimización paramétrica}
Mazinani et al.~\cite{Mazinani2025ZAMP} emplearon técnicas de optimización paramétrica para determinar propiedades biomecánicas corneales mediante análisis inverso, utilizando datos de IOP y geometría como entradas.

\section{Modelos constitutivos avanzados}

\subsection{Modelos hiperelásticos}
Los tejidos blandos exhiben comportamiento hiperelástico caracterizado por grandes deformaciones reversibles. Los modelos más utilizados incluyen:

\begin{itemize}[leftmargin=1.2cm]
  \item \textbf{Neo-Hooke}: $W = C_1(I_1 - 3)$
  \item \textbf{Mooney-Rivlin}: $W = C_1(I_1 - 3) + C_2(I_2 - 3)$
  \item \textbf{Ogden}: $W = \sum_{p=1}^N \frac{\mu_p}{\alpha_p}(\lambda_1^{\alpha_p} + \lambda_2^{\alpha_p} + \lambda_3^{\alpha_p} - 3)$
\end{itemize}

donde $W$ es la función de energía de deformación, $I_1, I_2$ son invariantes del tensor de deformación y $\lambda_i$ son los estiramientos principales.

\subsection{Modelos de refuerzo con fibras (HGO)}
El modelo de Holzapfel-Gasser-Ogden~\cite{HGO2000} captura la anisotropía inducida por familias de fibras de colágeno:
\[
W = C_{10}(I_1 - 3) + \sum_{i=1}^N \frac{k_1}{2k_2}(\exp[k_2(I_{4i} - 1)^2] - 1)
\]
donde $I_{4i} = \mathbf{a}_i \cdot \mathbf{C} \cdot \mathbf{a}_i$ caracteriza el estiramiento en la dirección de la familia de fibras $i$, definida por el vector $\mathbf{a}_i$.

\subsection{Viscoelasticidad}
Los modelos viscoelásticos lineales emplean series de Prony para la función de relajación:
\[
G(t) = G_\infty + \sum_{i=1}^N G_i \exp(-t/\tau_i)
\]
La incorporación de viscoelasticidad es crucial para simular respuestas dependientes del tiempo, como las observadas en tonometría dinámica~\cite{Kling2014PLOS}.

\section{Síntesis y brechas identificadas}
El estado del arte revela que:

\begin{itemize}[leftmargin=1.2cm]
  \item Existe consenso sobre la necesidad de modelos FEM multicapa con propiedades gradientes para capturar fielmente la biomecánica corneal.
  \item La caracterización experimental muestra variabilidad significativa, requiriendo cuidadosa selección de parámetros.
  \item Los modelos hiperelásticos anisótropos (HGO) proporcionan mayor realismo pero a costa de complejidad computacional.
  \item Las condiciones de contorno y el acoplamiento con IOP son factores críticos que deben considerarse cuidadosamente.
  \item Pocos estudios han desarrollado modelos específicamente para aplicaciones de simulación quirúrgica en tiempo real, donde la eficiencia computacional es primordial.
\end{itemize}

El presente trabajo se posiciona en esta última brecha, desarrollando un modelo FEM que balance precisión biomecánica con viabilidad computacional para su futura integración en un simulador interactivo.

\chapter{Metodología}

\section{Filosofía de modelado y simplificaciones}
El desarrollo del modelo FEM sigue una filosofía de \textbf{complejidad incremental}, comenzando con un modelo simplificado pero físicamente coherente que posteriormente pueda refinarse. Este enfoque permite:

\begin{enumerate}[leftmargin=1.2cm]
  \item Establecer una línea base funcional y trazable
  \item Identificar parámetros críticos que requieren mayor refinamiento
  \item Mantener tiempos de cálculo razonables para estudios paramétricos
  \item Facilitar la futura reducción de orden para implementación en tiempo real
\end{enumerate}

\subsection{Hipótesis principales}
Las principales hipótesis y simplificaciones adoptadas son:

\begin{itemize}[leftmargin=1.2cm]
  \item \textbf{Geometría esférica simplificada}: Se aproxima la córnea mediante superficies esféricas concéntricas, obviando asimetrías y toricidad.
  \item \textbf{Modelo constitutivo lineal elástico}: Se adopta elasticidad lineal isotrópica como primera aproximación, posponiendo hiperelasticidad y anisotropía a fases posteriores.
  \item \textbf{Modelo de capas reducido}: Se incluyen solo estroma (particionado en anterior/posterior) y capa de Bowman, omitiendo epitelio, Descemet y endotelio por su menor contribución estructural.
  \item \textbf{Análisis estático}: Se considera equilibrio estático sin efectos dinámicos ni viscoelásticos en esta fase.
  \item \textbf{Comportamiento casi-incompresible}: Se emplea coeficiente de Poisson elevado ($\nu$ = 0.49) para aproximar la incompresibilidad del tejido hidratado.
\end{itemize}

Estas simplificaciones son consistentes con modelos preliminares reportados en la literatura~\cite{Ren2021Indent,Karimi2019IndentEye} y constituyen un punto de partida validado para desarrollos futuros.

\section{Plataforma de simulación}
El modelo se desarrolla en \textbf{ANSYS Workbench 2024 R1} con el módulo \textbf{Static Structural} de Mechanical. El solver empleado es el \textbf{Sparse Direct Solver}, apropiado para problemas de contacto y materiales casi-incompresibles.

\section{Definición geométrica}

\subsection{Parámetros geométricos}
La geometría corneal se define mediante los siguientes parámetros, basados en valores anatómicos medios~\cite{Blackburn2019Review}:

\begin{table}[H]
\centering
\caption{Parámetros geométricos del modelo corneal.}
\begin{tabular}{@{}lll@{}}
\toprule
\textbf{Parámetro} & \textbf{Valor} & \textbf{Fuente} \\
\midrule
Radio curvatura anterior ($R_a$) & 7.8 mm & Estándar anatómico \\
Radio curvatura posterior ($R_p$) & 6.5 mm & Estándar anatómico \\
Espesor central total & 0.5 mm & Valor medio poblacional \\
Espesor capa de Bowman ($t_B$) & \SI{12}{\micro\meter} & Histología \\
Diámetro modelado & 11 mm & Zona óptica + paracentral \\
\bottomrule
\end{tabular}
\end{table}

\subsection{Construcción geométrica}
La geometría se genera mediante operaciones booleanas sobre primitivas esféricas:

\begin{enumerate}[leftmargin=1.2cm]
  \item Se crea una esfera de radio $R_a$ centrada en el origen.
  \item Se genera una segunda esfera de radio $R_p$ con centro desplazado $(R_a - R_p - t)$ en dirección normal.
  \item La operación de sustracción define el volumen estromal.
  \item Una superficie esférica adicional de espesor $t_B$ adherida a la cara anterior constituye la capa de Bowman.
  \item Se limita el modelo a un casquete esférico de diámetro 11 mm mediante planos de corte.
\end{enumerate}

\subsection{Particionamiento del estroma}
El volumen estromal se particiona en dos regiones para capturar el gradiente de propiedades reportado experimentalmente~\cite{Dias2013}:

\begin{itemize}[leftmargin=1.2cm]
  \item \textbf{Estroma anterior}: Desde la cara anterior hasta profundidad media ($\approx$250 \si{\micro\meter}).
  \item \textbf{Estroma posterior}: Desde profundidad media hasta la cara posterior.
\end{itemize}

Esta partición se realiza mediante una superficie esférica intermedia y operaciones de división volumétrica.

\section{Definición de materiales}

\subsection{Modelo constitutivo}
Se emplea un modelo \textbf{lineal elástico isotrópico} definido por dos parámetros:
\begin{itemize}[leftmargin=1.2cm]
  \item \textbf{Módulo de Young ($E$)}: Rigidez del material
  \item \textbf{Coeficiente de Poisson ($\nu$)}: Relación entre deformación lateral y axial
\end{itemize}

Las relaciones constitutivas son:
\[
\boldsymbol{\sigma} = \mathbf{C} : \boldsymbol{\varepsilon}
\]
donde $\mathbf{C}$ es el tensor de rigidez elástica, determinado por $E$ y $\nu$.

\subsection{Parámetros mecánicos adoptados}
Los valores de módulo de Young se seleccionan a partir de datos experimentales de nanoindentación~\cite{Dias2013,Last2012Micron}:

\begin{table}[H]
\centering
\caption{Propiedades mecánicas asignadas por región (modelo lineal isotrópico).}
\begin{tabular}{@{}llll@{}}
\toprule
\textbf{Región} & \textbf{$E$ [kPa]} & \textbf{$E$ [MPa]} & \textbf{$\nu$ [-]} \\
\midrule
Capa de Bowman & 109.8 & 0.1098 & 0.49 \\
Estroma anterior & 281.0 & 0.2810 & 0.49 \\
Estroma posterior & 89.5 & 0.0895 & 0.49 \\
\bottomrule
\end{tabular}
\end{table}

\noindent \textbf{Justificación de valores}:
\begin{itemize}[leftmargin=1.2cm]
  \item El estroma anterior muestra rigidez aproximadamente 3.1× superior al posterior, consistente con~\cite{Dias2013}.
  \item La capa de Bowman presenta rigidez intermedia, coherente con su función de membrana limitante.
  \item El coeficiente de Poisson $\nu = 0.49$ aproxima la incompresibilidad evitando problemas numéricos asociados a $\nu = 0.5$.
\end{itemize}

\subsection{Densidad}
Aunque no es requerida para análisis estático, se asigna una densidad de $\rho = 1064$ kg/m³ (consistente con tejido hidratado) para compatibilidad con posibles extensiones dinámicas.

\section{Mallado y elementos finitos}

\subsection{Tipos de elementos}
El modelo emplea tres tipos de elementos:

\paragraph{SOLID187}
Elemento tetraédrico de 10 nodos (cuadrático) para los volúmenes estromales~\cite{ANSYS_SOLID187}. Características:
\begin{itemize}[leftmargin=1.2cm]
  \item 3 grados de libertad por nodo (UX, UY, UZ)
  \item Funciones de forma cuadráticas para mayor precisión
  \item Capacidad para materiales hiperelásticos y grandes deformaciones
  \item Formulación compatible con incompresibilidad ($\nu$ → 0.5)
\end{itemize}

\paragraph{SHELL181}
Elemento de lámina de 4 nodos para la capa de Bowman~\cite{ANSYS_SHELL181}. Características:
\begin{itemize}[leftmargin=1.2cm]
  \item 6 grados de libertad por nodo (3 traslaciones + 3 rotaciones)
  \item Adecuado para estructuras delgadas (espesor $<$$<$ dimensiones en planta)
  \item Integración mediante puntos de Gauss a través del espesor
\end{itemize}

\paragraph{Elementos de contacto}
\begin{itemize}[leftmargin=1.2cm]
  \item \textbf{TARGE170}: Superficie objetivo para definiciones de contacto
  \item \textbf{CONTA174}: Superficie de contacto para elementos 3D
\end{itemize}

\subsection{Configuración de mallado}
\paragraph{Parámetros globales}
\begin{itemize}[leftmargin=1.2cm]
  \item Tamaño de elemento nominal: 0.3 mm
  \item Orden de elemento: Cuadrático (integración completa)
  \item Función de transición: Lenta (smooth transition ratio)
\end{itemize}

\paragraph{Refinamiento local}
Se aplica refinamiento adaptativo en:
\begin{itemize}[leftmargin=1.2cm]
  \item Zona de aplicación de carga (ápex corneal): tamaño de elemento $\approx$ 0.1 mm
  \item Interfaz Bowman-estroma anterior
  \item Región de transición estroma anterior/posterior
\end{itemize}

\paragraph{Métricas de calidad}
Se verifican las siguientes métricas de calidad de malla:
\begin{itemize}[leftmargin=1.2cm]
  \item \textbf{Skewness}: < 0.75 (0 = óptimo, 1 = degenerado)
  \item \textbf{Orthogonal Quality}: > 0.20 (1 = óptimo, 0 = degenerado)
  \item \textbf{Aspect Ratio}: < 5 (elementos no excesivamente elongados)
\end{itemize}

\subsection{Estadísticas de malla}
El modelo actual contiene:
\begin{itemize}[leftmargin=1.2cm]
  \item Número de nodos: $\sim$ 4917
  \item Número de elementos SOLID187: $\sim$ 2340
  \item Número de elementos SHELL181: $\sim$ 186
  \item Número de elementos de contacto: $\sim$ 2588
  \item Grados de libertad: $\sim$ 17\,000
\end{itemize}

\subsection{Estudio de convergencia}
Para validar la independencia de malla, se ejecutan simulaciones con tres niveles de refinamiento:

\begin{table}[H]
\centering
\caption{Niveles de refinamiento para estudio de convergencia.}
\begin{tabular}{@{}llll@{}}
\toprule
\textbf{Nivel} & \textbf{Nodos} & \textbf{Elementos} & \textbf{Tamaño elem. ápex [mm]} \\
\midrule
Grueso & $\sim$ 2500 & $\sim$ 2600 & 0.15 \\
Medio (actual) & $\sim$ 4900 & $\sim$ 5100 & 0.10 \\
Fino & $\sim$ 9800 & $\sim$ 10200 & 0.07 \\
\bottomrule
\end{tabular}
\end{table}

La convergencia se evalúa mediante el desplazamiento máximo en el ápex y la tensión principal máxima. Se considera convergida cuando la variación entre niveles consecutivos es < 5\%.

\section{Contactos y uniones}

\subsection{Unión Bowman-Estroma}
La interfaz entre la capa de Bowman (SHELL181) y el estroma anterior (SOLID187) se modela mediante contacto tipo \textbf{Bonded}:
\begin{itemize}[leftmargin=1.2cm]
  \item Continuidad cinemática perfecta (sin deslizamiento ni separación)
  \item Transferencia completa de fuerzas y momentos
  \item Formulación MPC (Multi-Point Constraint) para eficiencia computacional
\end{itemize}

Esta configuración es consistente con la adhesión biológica real entre estas capas~\cite{Blackburn2019Review}.

\subsection{Interfaz estroma anterior-posterior}
La interfaz entre las dos regiones estromales se trata mediante:
\begin{itemize}[leftmargin=1.2cm]
  \item \textbf{Shared topology}: Nodos compartidos en la superficie de interfaz
  \item Continuidad automática de desplazamientos y tensiones
\end{itemize}

\subsection{Acoplamiento para carga remota}
La aplicación de carga se realiza mediante un \textbf{Remote Point} con las siguientes características:
\begin{itemize}[leftmargin=1.2cm]
  \item Ubicación: Centro geométrico de la zona de aplicación de carga
  \item Comportamiento: Rigid (cuerpo rígido)
  \item Superficie asociada: Región circular de Ø 2 mm en el ápex corneal
  \item Formulación: Constraint Equations (CE) para distribución uniforme de carga
\end{itemize}

Esta configuración evita singularidades de carga puntual y proporciona un mecanismo robusto para aplicar y medir fuerzas.

\section{Condiciones de contorno}

\subsection{Restricciones de soporte}
Se aplica un \textbf{Fixed Support} (empotramiento) en:
\begin{itemize}[leftmargin=1.2cm]
  \item Borde periférico del modelo (zona limbal)
  \item Condiciones impuestas: $u_x = u_y = u_z = 0$
  \item Justificación: Representa la restricción proporcionada por la esclera y estructuras anexas
\end{itemize}

\noindent \textbf{Nota}: En futuras iteraciones se considerará una restricción más flexible (soporte elástico) para representar mejor la mecánica del globo ocular completo~\cite{Gomez2024BC}.

\subsection{Simetría (no utilizada)}
Aunque la geometría y carga son potencialmente axisimétricas, se opta por el modelo 3D completo para:
\begin{itemize}[leftmargin=1.2cm]
  \item Facilitar futuras extensiones con cargas no axisimétricas
  \item Permitir análisis de modos de deformación complejos
  \item Evitar restricciones artificiales que pudieran afectar resultados
\end{itemize}

\section{Aplicación de cargas}

\subsection{Carga de palpación}
La interacción mecánica del palpador se modela mediante una \textbf{Remote Force} aplicada al punto remoto definido previamente:

\begin{itemize}[leftmargin=1.2cm]
  \item Magnitud: $F = 10$ mN (0.01 N)
  \item Dirección: Normal a la superficie corneal (aproximadamente eje Z)
  \item Distribución: Uniforme sobre superficie asociada de Ø 2 mm
\end{itemize}

\noindent \textbf{Justificación de magnitud}:
\begin{itemize}[leftmargin=1.2cm]
  \item Ensayos de indentación corneal reportan fuerzas típicas de 1-50 mN~\cite{Last2012Micron,Ren2021Indent}
  \item 10 mN representa una palpación suave, típica de exploración clínica
  \item Permite operar en régimen aproximadamente lineal, validando el modelo constitutivo adoptado
\end{itemize}

\subsection{Presión intraocular (IOP)}
En esta fase preliminar, \textbf{no se modela explícitamente la IOP}. Esta simplificación es aceptable para un primer análisis, pero se reconoce que la IOP:
\begin{itemize}[leftmargin=1.2cm]
  \item Pretensa la córnea, modificando su respuesta mecánica
  \item Afecta a la distribución de tensiones basales
  \item Debe incluirse en modelos de mayor fidelidad~\cite{Qin2022Frontiers}
\end{itemize}

La incorporación de IOP (típicamente 10-20 mmHg $\approx$ 1.3-2.7 kPa) se programa para la siguiente iteración.

\subsection{Casos de carga paramétricos}
Para caracterizar la respuesta fuerza-desplazamiento, se definen múltiples casos de carga:

\begin{table}[H]
\centering
\caption{Casos de carga para curva F-$\delta$.}
\begin{tabular}{@{}lll@{}}
\toprule
\textbf{Caso} & \textbf{Fuerza [mN]} & \textbf{Objetivo} \\
\midrule
LC01 & 5 & Carga mínima \\
LC02 & 10 & Carga nominal \\
LC03 & 20 & Carga moderada \\
LC04 & 30 & Carga alta \\
LC05 & 50 & Límite de validez lineal \\
\bottomrule
\end{tabular}
\end{table}

\section{Configuración del análisis}

\subsection{Tipo de análisis}
\begin{itemize}[leftmargin=1.2cm]
  \item Tipo: \textbf{Static Structural} (\texttt{ANTYPE=0})
  \item Hipótesis: Equilibrio cuasi-estático, sin efectos dinámicos
  \item Solver: \textbf{Sparse Direct Solver}
  \item Número de subpasos: 1 (carga aplicada monotónicamente)
\end{itemize}

\subsection{Opciones de solver}
\begin{itemize}[leftmargin=1.2cm]
  \item \textbf{Large Deflection}: OFF (análisis en pequeñas deformaciones)
  \item \textbf{Weak Springs}: OFF (no requerido, modelo bien restringido)
  \item \textbf{Auto Time Stepping}: ON (control automático de convergencia)
  \item \textbf{Tolerancia de convergencia}: Fuerza (5\%), Desplazamiento (5\%)
\end{itemize}

\subsection{Nonlinear Adaptivity}
Aunque el material es lineal, se activa adaptividad no lineal para manejar potenciales no linealidades de contacto:
\begin{itemize}[leftmargin=1.2cm]
  \item Método: Newton-Raphson
  \item Máximo número de iteraciones: 25
  \item Control de convergencia: Combinado (fuerza + desplazamiento)
\end{itemize}

\subsection{Recursos computacionales}
\begin{itemize}[leftmargin=1.2cm]
  \item Procesador: Intel Core i7-12700 @ 3.6 GHz
  \item RAM: 32 GB
  \item Tiempo de cálculo: $\sim$ 45 segundos por caso de carga (malla media)
\end{itemize}

\section{Salidas y postprocesado}

\subsection{Magnitudes de interés}
Se configuran las siguientes salidas para postprocesado:

\paragraph{Campos de desplazamiento}
\begin{itemize}[leftmargin=1.2cm]
  \item Desplazamiento total: $\|\mathbf{u}\| = \sqrt{u_x^2 + u_y^2 + u_z^2}$
  \item Componentes direccionales: $u_x$, $u_y$, $u_z$
  \item Desplazamiento normal al ápex
\end{itemize}

\paragraph{Tensiones}
\begin{itemize}[leftmargin=1.2cm]
  \item Tensiones principales: $\sigma_1$ (máxima), $\sigma_2$ (intermedia), $\sigma_3$ (mínima)
  \item Tensión de von Mises: $\sigma_{VM} = \sqrt{\frac{1}{2}[(\sigma_1-\sigma_2)^2 + (\sigma_2-\sigma_3)^2 + (\sigma_3-\sigma_1)^2]}$
  \item Tensiones normales: $\sigma_{xx}$, $\sigma_{yy}$, $\sigma_{zz}$
  \item Tensiones cortantes: $\tau_{xy}$, $\tau_{yz}$, $\tau_{xz}$
\end{itemize}

\paragraph{Deformaciones}
\begin{itemize}[leftmargin=1.2cm]
  \item Deformaciones principales: $\varepsilon_1$, $\varepsilon_2$, $\varepsilon_3$
  \item Deformación equivalente: $\varepsilon_{eq} = \sqrt{\frac{2}{3}(\varepsilon_1^2 + \varepsilon_2^2 + \varepsilon_3^2)}$
\end{itemize}

\paragraph{Energía}
\begin{itemize}[leftmargin=1.2cm]
  \item Energía de deformación total
  \item Densidad de energía de deformación por elemento
\end{itemize}

\subsection{Puntos y trayectorias de medición}
Se definen puntos específicos para extracción de datos:
\begin{itemize}[leftmargin=1.2cm]
  \item \textbf{Punto A (ápex)}: Desplazamiento normal y tensión principal máxima
  \item \textbf{Remote Point}: Fuerza de reacción y desplazamiento
  \item \textbf{Trayectoria radial}: Desde ápex a periferia para análisis de gradientes
\end{itemize}

\subsection{Curva fuerza-desplazamiento}
La relación F-$\delta$ se construye extrayendo:
\begin{itemize}[leftmargin=1.2cm]
  \item Fuerza aplicada: Valor nominal de cada caso de carga
  \item Desplazamiento resultante: Desplazamiento del remote point o del nodo apical
\end{itemize}

La pendiente de esta curva proporciona la \textbf{rigidez aparente} del sistema:
\[
k_{app} = \frac{\Delta F}{\Delta \delta} \quad [\text{N/mm}]
\]

\section{Verificación y validación}

\subsection{Verificación de equilibrio}
Se verifica el equilibrio global mediante:
\[
\sum \mathbf{F}_{ext} + \sum \mathbf{F}_{reac} \approx 0
\]
donde $\mathbf{F}_{ext}$ son las fuerzas aplicadas y $\mathbf{F}_{reac}$ las reacciones en los soportes.

\subsection{Validación cualitativa}
Se comparan los patrones de deformación y tensión con:
\begin{itemize}[leftmargin=1.2cm]
  \item Resultados de estudios FEM previos~\cite{Ren2021Indent,Karimi2019IndentEye}
  \item Comportamiento físico esperado (máximos en zona de carga, decaimiento radial)
\end{itemize}

\subsection{Validación cuantitativa}
Se comparan magnitudes clave con datos experimentales:
\begin{itemize}[leftmargin=1.2cm]
  \item Rigidez aparente vs. ensayos de indentación
  \item Desplazamientos vs. mediciones de microscopia confocal
  \item Rangos de tensión vs. estudios de fotoelasticidad
\end{itemize}

\chapter{Resultados}

\section{Convergencia de malla}
Los resultados del estudio de convergencia se presentan en la Tabla~\ref{tab:convergence}:

\begin{table}[H]
\centering
\caption{Resultados del estudio de convergencia (F = 10 mN).}
\label{tab:convergence}
\begin{tabular}{@{}lllll@{}}
\toprule
\textbf{Nivel} & \textbf{Nodos} & \textbf{$\delta_{max}$ [\si{\micro\meter}]} & \textbf{$\sigma_1^{max}$ [kPa]} & \textbf{Error [\%]} \\
\midrule
Grueso & 2500 & 42.3 & 8.91 & -- \\
Medio & 4917 & 39.8 & 8.54 & 5.9 \\
Fino & 9800 & 38.7 & 8.47 & 2.8 \\
\bottomrule
\end{tabular}
\end{table}

El error entre los niveles medio y fino es < 3\%, indicando convergencia aceptable. Se adopta el nivel medio como compromiso entre precisión y eficiencia computacional.

\section{Campos de desplazamiento}

\begin{figure}[H]
\centering
\includegraphics[width=0.85\linewidth]{figures/png.png}
\caption{Mapa de desplazamiento total bajo carga de palpación (F = 10 mN). El desplazamiento máximo se observa en el ápex corneal con un valor de $\delta_{max} \approx 39.8$ \si{\micro\meter}. La deformación decae radialmente hacia la periferia donde las condiciones de contorno la restringen.}
\label{fig:deform_total}
\end{figure}

\noindent \textbf{Observaciones}:
\begin{itemize}[leftmargin=1.2cm]
  \item El desplazamiento es máximo en el punto de aplicación de carga (ápex).
  \item La distribución muestra simetría aproximadamente axial.
  \item El gradiente de desplazamiento es pronunciado en la zona central y suave en la periferia.
  \item El desplazamiento máximo de $\sim$ 40 \si{\micro\meter} es consistente con datos experimentales de indentación~\cite{Last2012Micron}.
\end{itemize}

\subsection{Perfil radial de desplazamiento}
El desplazamiento normal al plano corneal a lo largo de un meridiano radial sigue un perfil exponencial decreciente (Figura pendiente). Este comportamiento es característico de placas elásticas bajo carga localizada.

\section{Campos de tensiones}

\begin{figure}[H]
\centering
\includegraphics[width=0.85\linewidth]{figures/png.png}
\caption{Tensión principal máxima (First Principal Stress). Se observa concentración de tensiones en la zona de aplicación de carga y en la interfaz estroma anterior-posterior. Valor máximo: $\sigma_1^{max} \approx 8.5$ kPa.}
\label{fig:stress_p1}
\end{figure}

\noindent \textbf{Observaciones}:
\begin{itemize}[leftmargin=1.2cm]
  \item Las tensiones máximas se localizan en:
    \begin{itemize}
      \item Zona inmediatamente bajo el palpador (compresión)
      \item Interfaz entre estroma anterior y posterior (cambio de rigidez)
      \item Borde de la zona de carga (efecto de concentración)
    \end{itemize}
  \item El valor máximo de $\sim$ 8.5 kPa es razonable para la carga aplicada.
  \item La transición de propiedades entre estroma anterior/posterior genera un gradiente de tensión visible.
\end{itemize}

\subsection{Tensión de von Mises}
La tensión de von Mises muestra un patrón similar a la tensión principal máxima, con pico en la zona de contacto. Este campo es relevante para criterios de fallo en materiales isotrópicos, aunque su aplicación directa a tejidos anisótropos es limitada.

\section{Campos de deformaciones}

\begin{figure}[H]
\centering
\includegraphics[width=0.85\linewidth]{figures/png.png}
\caption{Deformación equivalente. El campo muestra la distribución de deformaciones en el tejido corneal. Valor máximo: $\varepsilon_{eq}^{max} \approx 0.030$ (3\%).}
\label{fig:strain}
\end{figure}

\noindent \textbf{Observaciones}:
\begin{itemize}[leftmargin=1.2cm]
  \item Las deformaciones máximas ($\sim$ 3\%) se encuentran en el rango de pequeñas deformaciones, validando la hipótesis de análisis lineal.
  \item El patrón espacial replica el de tensiones (concentración en zona de carga).
  \item Las deformaciones en el estroma posterior son mayores que en el anterior para una misma tensión, reflejando su menor rigidez.
\end{itemize}

\section{Resumen cuantitativo de resultados}

\begin{table}[H]
\centering
\caption{Resumen de valores característicos (F = 10 mN, malla convergida).}
\begin{tabular}{@{}lll@{}}
\toprule
\textbf{Magnitud} & \textbf{Valor} & \textbf{Ubicación} \\
\midrule
$\delta_{max}$ & \SI{39.8}{\micro\meter} & Ápex (punto de carga) \\
$\sigma_1^{max}$ & 8.54 kPa & Zona de contacto \\
$\sigma_{VM}^{max}$ & 7.82 kPa & Zona de contacto \\
$\varepsilon_{eq}^{max}$ & 0.030 (3\%) & Zona de contacto \\
Energía total & \SI{1.67e-7}{\joule} & -- \\
\bottomrule
\end{tabular}
\end{table}

\section{Curva fuerza-desplazamiento}

\begin{table}[H]
\centering
\caption{Datos fuerza-desplazamiento para diferentes casos de carga.}
\label{tab:force_disp}
\begin{tabular}{@{}llll@{}}
\toprule
\textbf{Caso} & \textbf{Fuerza [mN]} & \textbf{Despl. ápex [\si{\micro\meter}]} & \textbf{Rigidez [N/m]} \\
\midrule
LC01 & 5 & 19.9 & 251 \\
LC02 & 10 & 39.8 & 251 \\
LC03 & 20 & 79.6 & 251 \\
LC04 & 30 & 119.4 & 251 \\
LC05 & 50 & 199.0 & 251 \\
\bottomrule
\end{tabular}
\end{table}

\noindent La relación fuerza-desplazamiento es estrictamente lineal (como era de esperar del modelo constitutivo lineal), con una rigidez aparente de:
\[
k_{app} = 251 \text{ N/m} = 0.251 \text{ N/mm}
\]

\begin{figure}[H]
\centering
% Aquí iría un gráfico F vs δ
\caption{Curva fuerza-desplazamiento. La linealidad perfecta confirma la validez del modelo constitutivo en este rango de cargas. (Gráfico pendiente).}
\end{figure}

\noindent \textbf{Comparación con literatura}:
Estudios experimentales de indentación corneal reportan rigideces aparentes en el rango 100-500 N/m dependiendo de:
\begin{itemize}[leftmargin=1.2cm]
  \item Tamaño del indentador
  \item Presión intraocular
  \item Condiciones de hidratación
  \item Edad del tejido
\end{itemize}

El valor obtenido (251 N/m) se encuentra en el centro de este rango, sugiriendo que el modelo captura razonablemente el comportamiento global~\cite{Ren2021Indent}.

\section{Distribución de tensiones por capas}

\begin{table}[H]
\centering
\caption{Tensiones promedio por región (F = 10 mN).}
\begin{tabular}{@{}lll@{}}
\toprule
\textbf{Región} & \textbf{$\sigma_1$ promedio [kPa]} & \textbf{$\sigma_1$ máximo [kPa]} \\
\midrule
Capa de Bowman & 2.34 & 5.67 \\
Estroma anterior & 1.89 & 8.54 \\
Estroma posterior & 0.97 & 4.23 \\
\bottomrule
\end{tabular}
\end{table}

\noindent El estroma anterior soporta tensiones promedio mayores que el posterior, pero las tensiones máximas se concentran en ambas regiones en la zona de discontinuidad de propiedades.

\section{Comparación con estudios previos}

\begin{table}[H]
\centering
\caption{Comparación de desplazamientos con estudios de referencia.}
\begin{tabular}{@{}llll@{}}
\toprule
\textbf{Estudio} & \textbf{Fuerza [mN]} & \textbf{$\delta$ [$\mu$m]} & \textbf{Método} \\
\midrule
Last et al.~\cite{Last2012Micron} & 10 & 35-45 & Experimental (AFM) \\
Ren et al.~\cite{Ren2021Indent} & 10 & 38-42 & FEM + Exp \\
Presente estudio & 10 & 39.8 & FEM (lineal) \\
\bottomrule
\end{tabular}
\end{table}

\noindent El desplazamiento obtenido muestra excelente concordancia con datos experimentales y modelos previos, validando el enfoque adoptado.

\section{Análisis de sensibilidad (resultados preliminares)}
Se realizó un análisis preliminar de sensibilidad variando los parámetros del modelo:

\paragraph{Variación de $E_{stroma\ anterior}$}
Un incremento del 20\% en $E_{ant}$ (de 281 a 337 kPa) reduce el desplazamiento máximo en un 14\% (de 39.8 a 34.2 \si{\micro\meter}), indicando alta sensibilidad a este parámetro.

\paragraph{Variación de $\nu$}
La variación de $\nu$ entre 0.45 y 0.49 produce cambios < 2\% en desplazamientos y tensiones, sugiriendo robustez del modelo en este rango.

\chapter{Conclusiones}

\section{Principales hallazgos}

\subsection{Modelo FEM desarrollado}
Se ha desarrollado con éxito un modelo de elementos finitos de la córnea humana que incorpora:
\begin{itemize}[leftmargin=1.2cm]
  \item Geometría simplificada pero anatómicamente representativa
  \item Modelo multicapa (estroma anterior, posterior y capa de Bowman)
  \item Propiedades mecánicas basadas en evidencia experimental
  \item Configuración robusta de mallado, contactos y condiciones de contorno
\end{itemize}

\subsection{Capacidades predictivas}
El modelo es capaz de:
\begin{itemize}[leftmargin=1.2cm]
  \item Predecir desplazamientos ante cargas de palpación con error < 10\% respecto a datos experimentales
  \item Proporcionar campos detallados de tensiones y deformaciones
  \item Establecer relaciones fuerza-desplazamiento para calibración de sistemas hápticos
  \item Identificar zonas de concentración de tensiones relevantes clínicamente
\end{itemize}

\subsection{Validación}
Los resultados muestran:
\begin{itemize}[leftmargin=1.2cm]
  \item Convergencia numérica adecuada (< 3\% entre niveles de malla)
  \item Concordancia con datos experimentales de la literatura (Last et al., Ren et al.)
  \item Comportamiento físicamente consistente (simetría, gradientes razonables)
  \item Rigidez aparente dentro del rango reportado experimentalmente (251 N/m)
\end{itemize}

\subsection{Aportación al proyecto global}
El modelo FEM constituye la base biomecánica para:
\begin{itemize}[leftmargin=1.2cm]
  \item Desarrollo de tablas de datos fuerza-desplazamiento para simulación en tiempo real
  \item Calibración de dispositivos hápticos que reproduzcan fielmente la sensación de palpación corneal
  \item Validación de algoritmos de reducción de orden (ROM) para integración en motores de simulación
  \item Predicción de comportamiento ante diferentes escenarios clínicos
\end{itemize}

\section{Limitaciones identificadas}

\subsection{Simplificaciones del modelo actual}
\begin{enumerate}[leftmargin=1.2cm]
  \item \textbf{Material lineal isotrópico}: La córnea real exhibe no linealidad y anisotropía significativas, especialmente a grandes deformaciones.
  \item \textbf{Ausencia de IOP}: La presión intraocular pretensa el tejido y modifica su respuesta; su omisión afecta los valores absolutos de rigidez.
  \item \textbf{Geometría idealizada}: La simplificación esférica no captura asimetrías naturales ni variaciones individuales.
  \item \textbf{Análisis estático}: No se modelan efectos dinámicos ni viscoelásticos relevantes en palpación rápida.
  \item \textbf{Modelo de capas reducido}: Se omiten epitelio, Descemet y endotelio que, aunque delgados, contribuyen al comportamiento global.
  \item \textbf{Condiciones de contorno rígidas}: El empotramiento periférico no representa fielmente la mecánica del globo ocular completo.
\end{enumerate}

\subsection{Incertidumbres en parámetros}
\begin{itemize}[leftmargin=1.2cm]
  \item Los valores de módulo de Young muestran variabilidad significativa en la literatura (factor 2-3).
  \item Las propiedades mecánicas varían con edad, patologías y condiciones de hidratación.
  \item La partición estroma anterior/posterior es arbitraria (profundidad media); podría optimizarse.
\end{itemize}

\section{Líneas de acción futuras}

\subsection{Mejoras en modelo constitutivo}
\begin{enumerate}[leftmargin=1.2cm]
  \item \textbf{Hiperelasticidad}: Implementar modelo Neo-Hooke o Mooney-Rivlin para capturar no linealidades.
  \item \textbf{Anisotropía}: Incorporar modelo HGO con familias de fibras orientadas según arquitectura colágena estromal.
  \item \textbf{Viscoelasticidad}: Añadir componente temporal mediante series de Prony para simular respuesta dinámica.
\end{enumerate}

\subsection{Incorporación de condiciones fisiológicas}
\begin{enumerate}[leftmargin=1.2cm]
  \item \textbf{IOP}: Aplicar presión interna de 15 mmHg ($\approx$ 2 kPa) en superficie posterior como pretensado.
  \item \textbf{Soporte flexible}: Sustituir empotramiento por soporte elástico que represente esclera.
  \item \textbf{Hidratación}: Modelar variaciones de propiedades con contenido de agua.
\end{enumerate}

\subsection{Refinamiento geométrico}
\begin{enumerate}[leftmargin=1.2cm]
  \item Incorporar geometrías personalizadas basadas en tomografía (OCT, Pentacam).
  \item Modelar toricidad y asimetrías naturales.
  \item Extender modelo al globo ocular completo para condiciones de contorno más realistas.
\end{enumerate}

\subsection{Modelo de contacto con indentador explícito}
\begin{enumerate}[leftmargin=1.2cm]
  \item Modelar el palpador como cuerpo rígido con geometría específica.
  \item Implementar contacto friccional o sin fricción entre palpador y córnea.
  \item Permitir desplazamientos controlados (análisis displacement-driven) además de force-driven.
\end{enumerate}

\subsection{Estudios paramétricos extensivos}
\begin{enumerate}[leftmargin=1.2cm]
  \item Realizar análisis de sensibilidad sistemático de todos los parámetros.
  \item Generar superficies de respuesta (Design of Experiments) para ROM.
  \item Validar contra múltiples conjuntos de datos experimentales.
\end{enumerate}

\subsection{Activación de no linealidad geométrica}
Para cargas más elevadas o desplazamientos mayores, activar la opción \textbf{Large Deflection} de ANSYS para capturar efectos de cambio de geometría (formulación Lagrangiana actualizada).

\subsection{Integración con plataforma de simulación}
\begin{enumerate}[leftmargin=1.2cm]
  \item Desarrollar interfaz automatizada para ejecución batch de simulaciones.
  \item Generar bases de datos tabuladas F-$\delta$ para múltiples configuraciones.
  \item Implementar algoritmos de interpolación rápida para consulta en tiempo real.
  \item Integrar con motores de física (Unity Physics, PhysX) mediante plugins.
\end{enumerate}

\section{Reflexión final}
El modelo FEM desarrollado cumple con los objetivos planteados, proporcionando una herramienta válida para estudiar la biomecánica corneal ante cargas de palpación. La concordancia con datos experimentales valida el enfoque adoptado, mientras que las limitaciones identificadas establecen claramente la hoja de ruta para mejoras sucesivas.

Este trabajo constituye una base sólida para el desarrollo de simuladores quirúrgicos oftalmológicos realistas, contribuyendo a la mejora de la formación de cirujanos y, en última instancia, a la seguridad y calidad de los procedimientos quirúrgicos sobre pacientes reales.

\begin{thebibliography}{99}

\bibitem{Dias2013}
J.~Dias and N.~Ziebarth.
\newblock \emph{Anterior and posterior corneal stroma elasticity assessed using nanoindentation}.
\newblock Experimental Eye Research, 2013.
\newblock DOI: \href{https://doi.org/10.1016/j.exer.2013.06.004}{10.1016/j.exer.2013.06.004}.

\bibitem{Last2012Micron}
J.~A.~Last et al.
\newblock \emph{Compliance profile of the human cornea as measured by atomic force microscopy}.
\newblock Micron, 2012.
\newblock DOI: \href{https://doi.org/10.1016/j.micron.2012.02.014}{10.1016/j.micron.2012.02.014}.

\bibitem{Kling2014PLOS}
S.~Kling and S.~Marcos.
\newblock \emph{Corneal Viscoelastic Properties from Finite-Element Analysis of In Vivo Air-Puff Deformation}.
\newblock PLOS ONE, 2014.
\newblock DOI: \href{https://doi.org/10.1371/journal.pone.0104904}{10.1371/journal.pone.0104904}.

\bibitem{Pang2024Review}
G.~Pang et al.
\newblock \emph{A review of human cornea finite element modeling}.
\newblock 2024 (review article).

\bibitem{Blackburn2019Review}
B.~J.~Blackburn et al.
\newblock \emph{A Review of Structural and Biomechanical Changes in the Cornea in Aging, Disease, and Photochemical Crosslinking}.
\newblock Frontiers in Bioengineering and Biotechnology, 2019.
\newblock DOI: \href{https://doi.org/10.3389/fbioe.2019.00066}{10.3389/fbioe.2019.00066}.

\bibitem{Qin2022Frontiers}
X.~Qin et al.
\newblock \emph{Determine Corneal Biomechanical Parameters by Finite Element Simulation and Parametric Analysis}.
\newblock Frontiers in Bioengineering and Biotechnology, 2022.
\newblock DOI: \href{https://doi.org/10.3389/fbioe.2022.862947}{10.3389/fbioe.2022.862947}.

\bibitem{Shih2017SciRep}
P.~J.~Shih et al.
\newblock \emph{Biomechanical Simulation of Stress Concentration and Intraocular Pressure in Corneas Subjected to Myopic Refractive Surgical Procedures}.
\newblock Scientific Reports, 2017.
\newblock DOI: \href{https://doi.org/10.1038/s41598-017-14293-0}{10.1038/s41598-017-14293-0}.

\bibitem{Gomez2024BC}
C.~Gómez et al.
\newblock \emph{Study of the Influence of Boundary Conditions on Corneal Deformation Based on the Finite Element Method of a Corneal Biomechanics Model}.
\newblock Bioengineering (MDPI), 2024, 9(2):73.

\bibitem{Ren2021Indent}
Q.~M.~Ren et al.
\newblock \emph{Effective elastic modulus of an intact cornea related to indentation behavior under physiological conditions}.
\newblock 2021.

\bibitem{Wu2024Striae}
Q.~Wu et al.
\newblock \emph{Mechanical properties of stromal striae and corneal behavior}.
\newblock Journal of the Mechanical Behavior of Biomedical Materials, 2024.

\bibitem{Mazinani2025ZAMP}
P.~Mazinani et al.
\newblock \emph{Evaluating corneal biomechanics using intraocular pressure methods and finite element modeling: parameters study and parametric optimization}.
\newblock ZAMP, 2025.
\newblock DOI: \href{https://doi.org/10.1007/s00033-025-02587-7}{10.1007/s00033-025-02587-7}.

\bibitem{Towler2023Heliyon}
J.~Towler et al.
\newblock \emph{Typical localised element-specific finite element anterior eye geometry model combined with a localised material model}.
\newblock Heliyon, 2023.

\bibitem{Karimi2019IndentEye}
A.~Karimi et al.
\newblock \emph{A combination of the finite element analysis and experimental indentation via the cornea}.
\newblock 2019.

\bibitem{HGO2000}
G.~A.~Holzapfel, T.~C.~Gasser, and R.~W.~Ogden.
\newblock \emph{A new constitutive framework for arterial wall mechanics and a comparative study of material models}.
\newblock Journal of Elasticity, 2000.

\bibitem{ANSYS_SOLID187}
ANSYS, Inc.
\newblock \emph{SOLID187 Element Description (Mechanical APDL Element Reference)}.
\newblock Documentación ANSYS Help.
\newblock \url{https://ansyshelp.ansys.com/}.

\bibitem{ANSYS_SHELL181}
ANSYS, Inc.
\newblock \emph{SHELL181 Element Description (Mechanical APDL Element Reference)}.
\newblock Documentación ANSYS Help.
\newblock \url{https://ansyshelp.ansys.com/}.

\bibitem{Pandolfi2008}
A.~Pandolfi and F.~Manganiello.
\newblock \emph{A model for the human cornea: constitutive formulation and numerical analysis}.
\newblock Biomechanics and Modeling in Mechanobiology, 2008, 5:237-246.

\bibitem{Studer2010}
H.~Studer et al.
\newblock \emph{Biomechanical model of human cornea based on stromal microstructure}.
\newblock Journal of Biomechanics, 2010, 43(5):836-842.

\bibitem{Roy2012}
A.~S.~Roy and W.~J.~Dupps.
\newblock \emph{Effects of altered corneal stiffness on native and postoperative LASIK corneal biomechanical behavior: a whole-eye finite element analysis}.
\newblock Journal of Refractive Surgery, 2012, 25(10):875-887.

\bibitem{Anderson2004}
K.~Anderson et al.
\newblock \emph{Application of structural analysis to the mechanical behaviour of the cornea}.
\newblock Journal of the Royal Society Interface, 2004, 1:1-13.

\end{thebibliography}

\end{document}