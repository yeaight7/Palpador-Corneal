\documentclass[12pt,a4paper]{report}

\usepackage[utf8]{inputenc}
\usepackage[T1]{fontenc}
\usepackage[spanish]{babel}
\usepackage{geometry}
\usepackage{graphicx}
\usepackage{float}
\usepackage{booktabs}
\usepackage{siunitx}
\usepackage{amsmath}
\usepackage{hyperref}
\usepackage{caption}
\usepackage{subcaption}
\usepackage{enumitem}
\usepackage{xcolor}

\geometry{margin=2.5cm}

\sisetup{
  output-decimal-marker = {.},
  per-mode = symbol,
  detect-all = true
}

\hypersetup{
  colorlinks=true,
  linkcolor=blue,
  citecolor=blue,
  urlcolor=blue
}

\title{\textbf{Memoria (Parte ANSYS)}\\
Simulación biomecánica de la córnea humana mediante FEM\\
\large Palpador corneal: modelado y análisis en ANSYS}
\author{Pablo Barceló, Paco Bustos, Javier Rivero}
\date{}

\begin{document}
\maketitle
\tableofcontents
\clearpage

\chapter*{Resumen}
\addcontentsline{toc}{chapter}{Resumen}
Este documento constituye una \textbf{memoria provisional} centrada exclusivamente en la parte de \textbf{modelado y simulación mediante Elementos Finitos (FEM) en ANSYS Workbench/Mechanical} para un palpador corneal. Se describe la geometría simplificada empleada, la definición de materiales, el mallado, las condiciones de contorno, la aplicación de carga y la configuración del solver. Se incluyen también los resultados previstos a reportar (con \emph{placeholders} para capturas), así como limitaciones y líneas de mejora para iteraciones posteriores.

\vspace{0.5cm}
\noindent\textbf{Nota:} La integración con Unity/sofá/lápiz háptico se documentará en otra memoria; aquí solo se cubre la parte ANSYS.

\chapter{Introducción}
El objetivo de esta parte del proyecto es desarrollar un \textbf{modelo FEM de córnea} que permita estimar la respuesta mecánica del tejido ante una interacción tipo palpación/indentación, de forma que:
\begin{itemize}[leftmargin=1.2cm]
  \item se obtengan campos de \textbf{desplazamiento}, \textbf{tensiones} y \textbf{deformaciones} bajo cargas controladas;
  \item se pueda derivar una relación \textbf{fuerza--desplazamiento} (rigidez aparente) útil para calibración y comparación experimental;
  \item el modelo sea lo suficientemente simple como para servir de base a una \textbf{reducción de orden} o aproximación (p.\,ej., superficie de respuesta) para ejecución en tiempo real en un simulador interactivo.
\end{itemize}

Dado que la córnea es un tejido blando, casi incompresible y con comportamiento complejo (anisotropía, viscoelasticidad, no linealidad), se adopta en esta fase un conjunto de \textbf{hipótesis y simplificaciones} con el fin de obtener un primer modelo funcional, trazable y ajustable.

\chapter{Hipótesis de partida y simplificaciones}
\section{Modelo de capas reducido}
La córnea humana presenta varias capas. En esta fase se adopta un modelo reducido con dos componentes principales:
\begin{itemize}[leftmargin=1.2cm]
  \item \textbf{Estroma}: representa el volumen principal y concentra la mayor contribución estructural.
  \item \textbf{Capa de Bowman}: modelada como una membrana/capa delgada unida al estroma anterior.
\end{itemize}
Se omiten epitelio, Descemet y endotelio por su espesor reducido y menor contribución en una aproximación macroestructural (fase provisional).

\section{Geometría simplificada}
La geometría se aproxima mediante superficies esféricas para las caras anterior y posterior, con espesores medios representativos. Esta idealización facilita el preprocesado, el mallado y la comparativa controlada de resultados.

\section{Modelo constitutivo}
Se adopta un \textbf{modelo lineal elástico isotrópico} con \textbf{coeficiente de Poisson alto} para aproximar la casi-incompresibilidad. Esta decisión permite una primera simulación reproducible; en iteraciones posteriores se considerará hiperelasticidad/viscoelasticidad y anisotropía.

\chapter{Metodología en ANSYS Workbench/Mechanical}

\section{Entorno de trabajo}
La simulación se ha desarrollado en ANSYS Workbench/Mechanical (MAPDL solver) con configuración de análisis \textbf{Static Structural}. El solver utilizado es el \textbf{Sparse Matrix Direct Solver} (configuración típica de Mechanical).

\section{Definición geométrica}
\subsection{Parámetros geométricos principales}
Se emplean los siguientes valores medios (modelo estándar):
\begin{itemize}[leftmargin=1.2cm]
  \item Radio de curvatura \textbf{anterior}: $R_a = \SI{7.8}{mm}$
  \item Radio de curvatura \textbf{posterior}: $R_p = \SI{6.5}{mm}$
  \item Espesor central total: $t \approx \SI{0.5}{mm}$
  \item Espesor capa de Bowman: $t_B \approx \SI{0.012}{mm}$ (\SI{12}{\micro m})
\end{itemize}

\subsection{Particionamiento por regiones}
Para capturar el gradiente biomecánico estromal, el estroma se divide en:
\begin{itemize}[leftmargin=1.2cm]
  \item \textbf{Estroma anterior} (\emph{Estroma Interior} en la estructura del modelo).
  \item \textbf{Estroma posterior} (\emph{Estroma Exterior} en la estructura del modelo).
\end{itemize}
La capa de Bowman se define como una capa superficial adherida al estroma anterior.

\section{Materiales}
\subsection{Suposición de Poisson casi incompresible}
Se adopta $\nu = 0.49$ para representar el carácter casi incompresible del tejido blando.

\subsection{Propiedades elásticas (fase provisional)}
Tabla resumen de materiales (valores en módulo de Young):
\begin{table}[H]
\centering
\caption{Propiedades mecánicas adoptadas (modelo lineal isotrópico, fase provisional).}
\begin{tabular}{@{}lll@{}}
\toprule
\textbf{Región} & \textbf{$E$} & \textbf{$\nu$} \\
\midrule
Capa de Bowman & \SI{109.8}{kPa} & 0.49 \\
Estroma anterior & \SI{281}{kPa} & 0.49 \\
Estroma posterior & \SI{89.5}{kPa} & 0.49 \\
\bottomrule
\end{tabular}
\end{table}

\noindent Para estabilidad numérica y coherencia con unidades en Mechanical, estos valores equivalen a:
\[
\SI{281}{kPa} = \SI{0.281}{MPa},\quad
\SI{89.5}{kPa} = \SI{0.0895}{MPa},\quad
\SI{109.8}{kPa} = \SI{0.1098}{MPa}.
\]

\subsection{Densidad (si aplica)}
Si se requiere densidad para extensiones dinámicas (no usada en estático lineal), se puede adoptar $\rho \approx \SI{1064}{kg/m^3}$ (orden de magnitud de tejido hidratado).

\section{Mallado y tipo de elemento}
\subsection{Elementos empleados}
El modelo utiliza:
\begin{itemize}[leftmargin=1.2cm]
  \item \textbf{SOLID187} para el volumen del estroma (tetraédrico de 10 nodos, comportamiento cuadrático).
  \item \textbf{SHELL181} para la capa de Bowman (elemento de lámina).
  \item Elementos de contacto \textbf{TARGE170/CONTA174} para uniones/ligaduras (p.\,ej., unión Bowman--estroma y acoplamientos tipo \emph{remote}).
\end{itemize}

\subsection{Estadísticas de malla (modelo actual)}
En la exportación actual del modelo:
\begin{itemize}[leftmargin=1.2cm]
  \item Número aproximado de nodos: $\sim 4917$
  \item Número aproximado de elementos: $\sim 5114$
\end{itemize}
Estas cifras son provisionales y deben verificarse/actualizarse si se remalla (especialmente si se refina en el ápex o zona de contacto).

\subsection{Recomendación de refinamiento}
Para un palpador/indentador, se recomienda:
\begin{itemize}[leftmargin=1.2cm]
  \item refinamiento local en zona central (ápex) donde se aplica la carga;
  \item control de calidad de tetraedros (skewness/orthogonal quality);
  \item estudio de convergencia (al menos 2--3 niveles de malla) sobre desplazamiento máximo y tensiones máximas.
\end{itemize}

\section{Contactos y uniones}
\subsection{Unión Bowman--Estroma}
Se considera una unión tipo \textbf{bonded} (continuidad cinemática) entre la lámina de Bowman y la superficie del estroma anterior, con objeto de transmitir deformaciones/tensiones sin deslizamiento relativo en esta fase.

\subsection{Acoplamientos para cargas remotas}
Se utiliza un \emph{Remote Point} (punto remoto interno) para aplicar la carga sobre un conjunto de nodos de una superficie definida, permitiendo distribuir la fuerza de manera controlada y evitando singularidades en un único nodo de la piel/superficie.

\section{Condiciones de contorno y cargas}
\subsection{Soporte fijo}
Se aplica un \textbf{Fixed Support} en el borde/nodos de referencia del modelo (región \texttt{\_FIXEDSU} en el preprocesado exportado), imponiendo desplazamientos nulos para simular una restricción equivalente al anclaje periférico.

\subsection{Carga de palpación (Remote Force)}
La interacción de palpación se modela aplicando una fuerza remota sobre una superficie (región \texttt{SUPERFICIEFUERZA}). En el estado actual del modelo la fuerza aplicada tiene magnitud:
\[
F \approx \SI{10}{mN} \quad (\text{componente principal en el eje normal del modelo}).
\]
\noindent Esta magnitud es consistente con cargas suaves de indentación/palpación en tejidos blandos; puede parametrizarse para generar curvas fuerza--desplazamiento.

\section{Configuración del análisis (Static Structural)}
\subsection{Tipo de análisis}
\begin{itemize}[leftmargin=1.2cm]
  \item Análisis: \textbf{estático} (\texttt{ANTYPE=0}).
  \item Solver: \textbf{directo esparso} (Sparse Direct).
  \item Número de subpasos: 1 (configuración actual).
\end{itemize}

\subsection{Grados de libertad}
El sistema resultante es del orden de $\sim 1.7\times 10^4$ ecuaciones (dependiendo de restricciones/contactos).

\section{Magnitudes de salida y postprocesado}
Para reportar resultados de forma consistente con biomecánica corneal y con el objetivo del proyecto, se consideran:
\begin{itemize}[leftmargin=1.2cm]
  \item \textbf{Desplazamiento total} ($\|\mathbf{u}\|$) y desplazamiento normal en el ápex.
  \item \textbf{Tensión principal máxima} (relevante para tejidos frágiles/fibrilares).
  \item \textbf{Deformación equivalente} o componentes principales.
  \item \textbf{Presión/contacto} (si se usa indentador con contacto real en futuras iteraciones).
  \item \textbf{Curva fuerza--desplazamiento}: lectura del desplazamiento del punto remoto (o ápex) frente a fuerza aplicada.
\end{itemize}

\chapter{Resultados preliminares (plantilla de reporte)}
\textbf{Importante:} Esta sección está preparada para incorporar capturas desde Mechanical. Sustituir los ficheros en \texttt{figures/} y actualizar pies de figura con valores máximos.

\section{Desplazamiento}
\begin{figure}[H]
\centering
\includegraphics[width=0.88\linewidth]{figures/png.png} % deform_total.png
\caption{Mapa de desplazamiento total bajo carga de palpación. (Insertar captura de Mechanical).}
\label{fig:deform_total}
\end{figure}

\section{Tensiones}
\begin{figure}[H]
\centering
\includegraphics[width=0.88\linewidth]{figures/png.png} % stress_principal1.png
\caption{Tensión principal máxima (First Principal Stress). (Insertar captura de Mechanical).}
\label{fig:stress_p1}
\end{figure}

\section{Deformaciones}
\begin{figure}[H]
\centering
\includegraphics[width=0.88\linewidth]{figures/png.png} % strain_eqv.png
\caption{Deformación equivalente/principal. (Insertar captura de Mechanical).}
\label{fig:strain}
\end{figure}

\section{Resumen numérico (rellenar)}
\begin{table}[H]
\centering
\caption{Resumen de valores máximos/mínimos (a completar tras postprocesado).}
\begin{tabular}{@{}lll@{}}
\toprule
\textbf{Métrica} & \textbf{Valor} & \textbf{Comentario} \\
\midrule
$\max \|\mathbf{u}\|$ & -- & Desplazamiento máximo \\
$u_\text{ápex}$ & -- & Desplazamiento normal en ápex \\
$\max \sigma_1$ & -- & Principal máxima \\
$\max \varepsilon$ & -- & Deformación representativa \\
\bottomrule
\end{tabular}
\end{table}

\chapter{Discusión, limitaciones y mejoras previstas}
\section{Limitaciones principales del modelo actual}
\begin{itemize}[leftmargin=1.2cm]
  \item \textbf{Material lineal isotrópico}: la córnea es anisótropa y no lineal; aquí se adopta una primera aproximación.
  \item \textbf{Ausencia de viscoelasticidad}: la respuesta real puede depender de la velocidad de palpación.
  \item \textbf{Condiciones fisiológicas simplificadas}: no se modela explícitamente la presión intraocular (IOP) ni el globo ocular completo.
  \item \textbf{Mallado provisional}: es necesario un estudio de convergencia y refinamiento local para tensiones.
\end{itemize}

\section{Mejoras recomendadas (siguiente iteración)}
\begin{itemize}[leftmargin=1.2cm]
  \item Activar y justificar \textbf{no linealidad geométrica} si se espera deformación apreciable.
  \item Incorporar \textbf{IOP} como presión interna equivalente (p.\,ej., \SIrange{10}{20}{mmHg}) para pre-tensar la córnea.
  \item Sustituir material lineal por \textbf{hiperelástico} (y opcionalmente anisótropo tipo HGO) y/o \textbf{viscoelástico}.
  \item Modelar el \textbf{indentador} explícitamente (rigid body + contacto friccional o sin fricción) en lugar de una fuerza remota, si se requiere realismo de interacción.
  \item Definir un conjunto de simulaciones paramétricas para construir una \textbf{superficie de respuesta} (ROM) destinada a Unity/háptica.
\end{itemize}

\chapter{Conclusiones (parte ANSYS)}
\begin{itemize}[leftmargin=1.2cm]
  \item Se ha construido un modelo FEM de córnea con un esquema reducido de capas (estroma + Bowman) y partición estromal anterior/posterior.
  \item Se ha definido un conjunto de propiedades elásticas provisionales y una configuración de análisis estático estructural.
  \item El modelo produce las magnitudes necesarias para el proyecto: desplazamientos, tensiones y la base para una curva fuerza--desplazamiento.
  \item Quedan como siguientes hitos la incorporación de condiciones fisiológicas (IOP), mejora constitutiva (hiperelasticidad/anisotropía), y validación mediante refinamiento y comparación bibliográfica/experimental.
\end{itemize}

\begin{thebibliography}{99}

\bibitem{Dias2013}
J.~Dias and N.~Ziebarth.
\newblock \emph{Anterior and posterior corneal stroma elasticity assessed using nanoindentation}.
\newblock Experimental Eye Research, 2013.
\newblock DOI: \href{https://doi.org/10.1016/j.exer.2013.06.004}{10.1016/j.exer.2013.06.004}.

\bibitem{Last2012Micron}
J.~A.~Last et al.
\newblock \emph{Compliance profile of the human cornea as measured by atomic force microscopy}.
\newblock Micron, 2012.
\newblock DOI: \href{https://doi.org/10.1016/j.micron.2012.02.014}{10.1016/j.micron.2012.02.014}.

\bibitem{Kling2014PLOS}
S.~Kling and S.~Marcos.
\newblock \emph{Corneal Viscoelastic Properties from Finite-Element Analysis of In Vivo Air-Puff Deformation}.
\newblock PLOS ONE, 2014.
\newblock DOI: \href{https://doi.org/10.1371/journal.pone.0104904}{10.1371/journal.pone.0104904}.

\bibitem{Pang2024Review}
G.~Pang et al.
\newblock \emph{A review of human cornea finite element modeling}.
\newblock 2024 (review).
\newblock (Consultar versión final del journal y completar volumen/páginas).

\bibitem{Blackburn2019Review}
B.~J.~Blackburn et al.
\newblock \emph{A Review of Structural and Biomechanical Changes in the Cornea in Aging, Disease, and Photochemical Crosslinking}.
\newblock Frontiers in Bioengineering and Biotechnology, 2019.
\newblock DOI: \href{https://doi.org/10.3389/fbioe.2019.00066}{10.3389/fbioe.2019.00066}.

\bibitem{Qin2022Frontiers}
X.~Qin et al.
\newblock \emph{Determine Corneal Biomechanical Parameters by Finite Element Simulation and Parametric Analysis}.
\newblock Frontiers in Bioengineering and Biotechnology, 2022.
\newblock DOI: \href{https://doi.org/10.3389/fbioe.2022.862947}{10.3389/fbioe.2022.862947}.

\bibitem{Shih2017SciRep}
P.~J.~Shih et al.
\newblock \emph{Biomechanical Simulation of Stress Concentration and Intraocular Pressure in Corneas Subjected to Myopic Refractive Surgical Procedures}.
\newblock Scientific Reports, 2017.
\newblock DOI: \href{https://doi.org/10.1038/s41598-017-14293-0}{10.1038/s41598-017-14293-0}.

\bibitem{Gomez2024BC}
C.~Gómez et al.
\newblock \emph{Study of the Influence of Boundary Conditions on Corneal Deformation Based on the Finite Element Method of a Corneal Biomechanics Model}.
\newblock Bioengineering (MDPI), 2024.
\newblock (Completar DOI exacto en versión final; artículo: 9(2):73).

\bibitem{Ren2021Indent}
Q.~M.~Ren et al.
\newblock \emph{Effective elastic modulus of an intact cornea related to indentation behavior under physiological conditions}.
\newblock 2021.
\newblock (Completar datos bibliográficos finales).

\bibitem{Wu2024Striae}
Q.~Wu et al.
\newblock \emph{Mechanical properties of stromal striae and corneal behavior}.
\newblock Journal of the Mechanical Behavior of Biomedical Materials, 2024.
\newblock (Completar volumen/páginas).

\bibitem{Mazinani2025ZAMP}
P.~Mazinani et al.
\newblock \emph{Evaluating corneal biomechanics using intraocular pressure methods and finite element modeling: parameters study and parametric optimization}.
\newblock ZAMP, 2025.
\newblock DOI: \href{https://doi.org/10.1007/s00033-025-02587-7}{10.1007/s00033-025-02587-7}.

\bibitem{Towler2023Heliyon}
J.~Towler et al.
\newblock \emph{Typical localised element-specific finite element anterior eye geometry model combined with a localised material model}.
\newblock Heliyon, 2023.
\newblock (Completar datos bibliográficos finales).

\bibitem{Karimi2019IndentEye}
A.~Karimi et al.
\newblock \emph{A combination of the finite element analysis and experimental indentation via the cornea}.
\newblock 2019.
\newblock (Completar journal/DOI según la versión consultada).

\bibitem{HGO2000}
G.~A.~Holzapfel, T.~C.~Gasser, and R.~W.~Ogden.
\newblock \emph{A new constitutive framework for arterial wall mechanics and a comparative study of material models}.
\newblock Journal of Elasticity, 2000.
\newblock (Usado como referencia para posible extensión anisótropa).

\bibitem{ANSYS_SOLID187}
ANSYS, Inc.
\newblock \emph{SOLID187 Element Description (Mechanical APDL Element Reference)}.
\newblock Documentación ANSYS Help.
\newblock \url{https://ansyshelp.ansys.com/}.

\bibitem{ANSYS_SHELL181}
ANSYS, Inc.
\newblock \emph{SHELL181 Element Description (Mechanical APDL Element Reference)}.
\newblock Documentación ANSYS Help.
\newblock \url{https://ansyshelp.ansys.com/}.

\bibitem{IndentationReviewASME2024}
\newblock \emph{A Comprehensive Review of Indentation of Gels and Soft Hydrated Materials}.
\newblock Applied Mechanics Reviews (ASME), 2024.
\newblock (Completar autores/DOI si se incluye en versión final).

\end{thebibliography}

\end{document}