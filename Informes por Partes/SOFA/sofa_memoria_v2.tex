\documentclass[12pt,a4paper]{report}

\usepackage[utf8]{inputenc}
\usepackage[T1]{fontenc}
\usepackage[spanish]{babel}

\usepackage{geometry}
\geometry{margin=2.5cm}

\usepackage{amsmath}
\usepackage{amssymb}
\usepackage{graphicx}
\usepackage{float}
\usepackage{booktabs}
\usepackage{siunitx}
\usepackage{enumitem}
\usepackage{xcolor}
\usepackage{hyperref}
\usepackage{caption}
\usepackage{subcaption}

\sisetup{
  output-decimal-marker = {.},
  per-mode = symbol,
  detect-all = true
}

\hypersetup{
  colorlinks=true,
  linkcolor=blue,
  citecolor=blue,
  urlcolor=blue
}

\title{Adaptación del modelo de córnea de ANSYS a SOFA\\
Versión 1: Construcción de la escena y modelado}
\author{Parte del trabajo grupal: Simulación SOFA}
\date{\today}

\begin{document}
\maketitle
\tableofcontents
\clearpage

\chapter*{Resumen}
\addcontentsline{toc}{chapter}{Resumen}
Este documento recoge la \textbf{parte de simulación en SOFA} del proyecto de palpación corneal. Se describe la construcción de la escena, la adaptación de la geometría y el mallado desde ANSYS hacia un modelo volumétrico compatible con SOFA (incluyendo el pipeline con Gmsh), la definición de capas corneales y propiedades mecánicas, y el diseño e integración del palpador. El objetivo es disponer de una escena reproducible y documentada como base para la integración posterior del simulador.

\clearpage

\chapter{Introducción y contexto}
\label{sec:intro}

El objetivo de esta parte del proyecto consiste en adaptar el modelo biomecánico de la córnea previamente modelada y simulada en ANSYS, de manera que pueda ejecutarse en el motor de simulación \textbf{SOFA} y servir como base para una futura integración en una interfaz interactiva en tiempo real. SOFA (Simulation Open Framework Architecture) es un framework de simulación física orientado a aplicaciones biomédicas, ampliamente utilizado en investigación para simulación de tejidos blandos y sistemas interactivos.

La motivación principal de esta adaptación es que, mientras ANSYS permite simulaciones detalladas y precisas de elementos finitos, no está pensado para la ejecución en tiempo real ni para interacción continua. SOFA, en cambio, permite simular deformación, contacto y fuerzas de forma interactiva, con posibilidad de conexión a dispositivos hápticos o integración con entornos 3D como Unity.

\chapter{Construcción de la escena SOFA}
\label{sec:escena}

\section{Software SOFA: características y estructura}
SOFA utiliza un \textbf{grafo de escena} donde cada nodo contiene componentes que definen tanto geometría como modelos físicos. Para simular tejidos blandos, SOFA requiere una topología de volumen (p.\,ej., elementos tetraédricos) y un conjunto de componentes que definan:
\begin{itemize}
  \item Grados de libertad (posición y velocidad de nodos).
  \item Modelo constitutivo (elasticidad, hiperelasticidad, etc.).
  \item Masa y fuerzas externas.
  \item Condiciones de contorno.
  \item Contacto y colisión.
\end{itemize}

\section{Problema 1: importación de mallas desde ANSYS}
En ANSYS se trabajó con una malla de elementos sólidos y/o capas, pero la exportación directa a SOFA presenta dificultades. ANSYS exporta normalmente en formatos de superficie (\texttt{.obj}, \texttt{.stl}) o de su propio entorno (\texttt{.cdb}). SOFA, sin embargo, requiere una malla volumétrica con conectividad explícita (tetraedros). Importar solo la superficie implica que no existe un volumen sobre el que calcular la elasticidad, y por tanto el modelo queda inconsistente, imposibilitando una simulación biomecánica realista.

\section{Solución 1: pipeline ANSYS → Gmsh → SOFA}
\label{sec:solucion_malla}

Se identificó que la solución era utilizar un software intermediario para reconstruir el volumen y generar la malla tetraédrica necesaria. Se utilizó \textbf{Gmsh}, herramienta de código abierto ampliamente utilizada en simulación numérica.

El pipeline implementado fue el siguiente:

\begin{enumerate}
  \item \textbf{Exportar desde ANSYS}: Se exportó el modelo de córnea como archivo \texttt{.obj} (formato de superficie).
  
  \item \textbf{Importar y procesar en Gmsh}: Se importó el archivo de superficie, se generó un volumen cerrado y se preparó el mallado.
  
  \item \textbf{Generación de malla 3D}: En Gmsh se definieron los parámetros de discretización (tamaño de elemento) y se ejecutó la generación de malla 3D. Gmsh genera automáticamente una malla de tetrahedros que rellena el volumen interior.
  
  \item \textbf{Exportar a formato .msh2}: Se exportó la malla generada en formato \texttt{.msh2} (ASCII 2). Este formato incluye:
    \begin{itemize}
      \item Coordenadas de todos los nodos.
      \item Conectividad de elementos (tetraedros): cada elemento se define por sus 4 nodos.
      \item Información de fronteras y etiquetado de regiones (grupos físicos).
    \end{itemize}
  
  \item \textbf{Importar en SOFA}: SOFA posee parsers nativos para \texttt{.msh2}, permitiendo cargar la topología tetraédrica y asignar automáticamente los grados de libertad y la conectividad del modelo.
\end{enumerate}

De esta manera se obtiene una representación volumétrica con topología explícita que SOFA puede utilizar para resolver la mecánica.

\chapter{Modelado de las capas cornéales}
\label{sec:modelado_capas}

\section{Estructura multicapa de la córnea}
Se buscó representar la córnea con un modelo multicapa análogo al desarrollado en ANSYS. En SOFA esto implica definir múltiples mallas o submallas con propiedades distintas. Las capas consideradas fueron:
\begin{itemize}
  \item \textbf{Estroma anterior}
  \item \textbf{Estroma posterior}
  \item \textbf{Capa de Bowman} (superficial)
\end{itemize}

Cada capa fue tratada como un objeto mecánico en SOFA, con su topología correspondiente y un modelo constitutivo asociado.

\section{Inserción de propiedades de material}
Para el estroma se usó un modelo de elasticidad lineal (FEM) con parámetros equivalentes a los usados en ANSYS, asignando:
\begin{itemize}
  \item Módulo de Young $E$ (según región anterior/posterior).
  \item Coeficiente de Poisson alto ($\nu \approx 0.49$) para aproximar incompresibilidad.
  \item Incorporar amortiguamiento viscoelástico mediante un parámetro de damping.
\end{itemize}

Para la \textbf{capa de Bowman} (membrana superficial):

\begin{itemize}
  \item Cargar la malla triangular \texttt{.msh2} correspondiente en un nodo del grafo de escena.
  \item Asignar grados de libertad mediante \texttt{MechanicalObject}.
  \item Definir un componente de fuerzas elásticas basado en elementos de membrana, introduciendo los parámetros mecánicos correspondientes.
  \item Incorporar amortiguamiento.
\end{itemize}

\section{Consideraciones sobre adimensionalidad en SOFA}
SOFA trabaja internamente con unidades SI, pero requiere coherencia en la escala. Se realizó una verificación de:
\begin{itemize}
  \item Unidades de longitud (m vs mm).
  \item Unidades de fuerza (N vs mN).
  \item Escalado de densidad y rigidez si procede.
\end{itemize}

Se decidió mantener la escala en metros para compatibilidad con el motor y evitar problemas numéricos, ajustando valores de geometría y parámetros en consecuencia.

\chapter{Diseño del palpador}
\label{sec:palpador}

\section{Especificación y modelado del palpador}
El palpador se modeló como un objeto rígido (o altamente rígido) con geometría simple (p.\,ej., esfera o cilindro con punta redondeada). Este componente se define con:
\begin{itemize}
  \item Malla de colisión (superficie).
  \item Estado mecánico (posición/orientación).
  \item Restricciones de movimiento (controlado por trayectoria o por input externo en futuras versiones).
\end{itemize}

\section{Integración en la escena SOFA}
La interacción palpador--córnea se implementa mediante:
\begin{itemize}
  \item Detección de colisión entre superficies.
  \item Modelo de contacto (penalización o constraints).
  \item Aplicación de fuerzas de reacción en la córnea.
\end{itemize}

En esta fase se configuró el contacto básico, dejando la parametrización fina (rigidez de contacto, fricción) para ajuste posterior.

\chapter{Escena SOFA resultante}
\label{sec:escena_final}

\section{Estructura de la escena .scn}
La escena final se organiza en nodos principales:
\begin{itemize}
  \item \textbf{Nodo Córnea}: contiene el modelo FEM y parámetros materiales.
  \item \textbf{Nodo Bowman}: capa superficial con propiedades de membrana.
  \item \textbf{Nodo Palpador}: objeto rígido con colisión.
  \item \textbf{Nodo Contacto}: configuración de detección y respuesta de contacto.
\end{itemize}

\section{Entregable para el equipo de Unity}
Como entregable se proporciona:
\begin{itemize}
  \item Archivo de escena \texttt{.scn} listo para ejecutar en SOFA.
  \item Mallas \texttt{.msh2} del volumen y de la capa de Bowman.
  \item Documentación del pipeline ANSYS $\to$ Gmsh $\to$ SOFA para reproducibilidad.
\end{itemize}

\chapter{Notas técnicas y consideraciones}
\label{sec:notas}

\section{Ventajas del pipeline implementado}
El pipeline permite:
\begin{itemize}
  \item Convertir geometrías superficiales de ANSYS a volúmenes tetraédricos aptos para FEM.
  \item Mantener consistencia geométrica entre plataformas.
  \item Reproducibilidad para futuras iteraciones con refinamientos de malla.
\end{itemize}

\section{Parámetros clave de la configuración}
Los parámetros más relevantes a ajustar en futuras versiones incluyen:
\begin{itemize}
  \item Tamaño de elemento en Gmsh (refinamiento local en el ápex corneal).
  \item Parámetros del solver en SOFA (time step, damping).
  \item Parámetros del modelo de contacto (rigidez de contacto, fricción, etc.).
  \item Validación con curvas fuerza--desplazamiento comparadas con ANSYS o bibliografía.
\end{itemize}

Se recomienda mantener un registro de cambios en la escena para facilitar la trazabilidad y futuras modificaciones.

\end{document}
